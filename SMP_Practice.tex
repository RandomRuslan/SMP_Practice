\documentclass{article}
\usepackage[english,russian]{babel}
\usepackage[14pt]{extsizes}
\usepackage[left=3cm,right=2cm,top=2cm,bottom=2cm,bindingoffset=0cm]{geometry}
\linespread{1.3}
\usepackage[dvips]{graphicx}
\usepackage{color}
\usepackage[colorlinks,linkcolor=blue,urlcolor=blue]{hyperref}
\usepackage[indentfirst]{titlesec}
\usepackage{titletoc}
\usepackage[utf8]{inputenc}
\setcounter{tocdepth}{3}
\author{Шаймарданов Р. Р.}
\begin{document}

\begin{titlepage}
	\begin{center}
		{\large\textsc{Университет ИТМО}}
		\vskip 1pt \hrule \vskip 3pt
		{\large\textsc{Кафедра вычислительной техники}}
		\vfill
		{\LargeОтчет по прохождению практики\vskip 12pt\bfseries Параллельные вычисления}	
	\end{center}
	\vfill	
	\begin{flushright}
		{Выполнил студент\\группы P3310\\Шаймарданов Р.\,Р.\vskip 12pt Руководитель\\Соснин В.\,В.}
	\end{flushright}	
	\vfill	
	\begin{center}
		Санкт-Петербург\\2016
	\end{center}
\end{titlepage}

\tableofcontents
\newpage

\section{Введение}
	\textbf{Цель:} составить составить список из наиболее авторитетных литературных источников по "Параллельным вычислениям".\\
	
	\textbf{Задачи}
	\begin{enumerate}
		\item Описать инструментарий, необходимый для выполнения практического задания.
   		\item Составить перечень современных (новее 2010 года) англоязычных источников, посвящённых параллельным вычислениям в системах с общей памятью.
   		\item Разбить найденные источники на четыре группы: 
   			\begin{itemize}
   		   		\item платные книги 
   		   		\item журналы 
   		   		\item бесплатные книги
   		   		\item онлайн-курсы
   		   	\end{itemize}
   		\item Сформулировать критерий уровня авторитетности для найденных материалов, выполнить ранжирование источников внутри групп.
   		\item Найти, провести разбиение и ранжировать русскоязычные источники, аналогично с иностранными источниками. 
	\end{enumerate}
\newpage
\section{Инструментарий}	
	\subsection{Система компьютерной верстки \TeX(\LaTeX)}
		\subsubsection{Описание}
	\TeX — это низкоуровневый язык разметки и программирования, созданный Дональдом Кнутом для единообразной вёрстки документов. Кнут начал разрабатывать систему набора текста \TeX в 1977 году для исследования потенциальных возможностей оборудования цифровой печати, которое в то время начинало проникать в издательское дело. Главным образом он надеялся улучшить качество печатной продукции, которое расстраивало в его собственных книгах и статьях. После выпуска в 1989 году поддержки восьмибитных символов разработка \TeX приостановилась, только иногда выходили версии с исправленными ошибками.\\

	\LaTeX  — основанный на \TeX пакет макросов, созданный Лесли Лампортом. Основная цель — упростить вёрстку текста, особенно в документах с математическими формулами. Значительно позднее авторы разработали для \LaTeX расширения, которые называются пакетами или стилями. Некоторые из них распространяются вместе с большинством дистрибутивов \TeX/\LaTeX.\\

	Так как \LaTeX содержит часть команд \TeX, то создание документа в \LaTeX - тоже программирование: создаётся текстовый файл в \LaTeX разметке, макросы LaTeX обрабатывают его и производят конечный документ.\\
		\subsubsection{Сравнение с другими программными средствами}
	Подход \LaTeX к созданию документа называется WYSIWYM\footnote{What You See Is What You Mean (То, что ты видишь, есть то, что ты имеешь в виду)}: во время набора текста Вы не видите окончательный вариант документа, только логическую структуру этого документа. Оформлением занимается сам \LaTeX. Такой подход имеет как достоинства, так и недостатки по сравнению с WYSIWYG\footnote{What You See Is What You Get (Что видишь, то и получишь)} программами, такими как Openoffice.org Writer или Microsoft Word.\\
	\textit{Достоинства:}
	\begin{itemize}
   		\item Файлы с исходными текстами можно просмотреть в любом текстовом редакторе, они понятнее в отличие от сложных бинарных файлов и форматов XML, используемых WYSIWYG программами.
   		\item Вы полностью сосредотачиваетесь на структуре и содержании документа и забываете о том, как будет выглядеть печатный вариант.
   		\item Не нужно вручную настраивать шрифты, размер текста, высоту строк или читаемость текста.
   		\item Легко скопировать структуру документа в другой документ, в WYSIWYG программах не всегда ясно, какое именно было использовано форматирование.
   		\item Разметка, шрифты, таблицы и т.д. согласованы во всём документе.
    	\itemЛегко набирать математические формулы.
    	\itemЛегко создаются алфавитные указатели, сноски, ссылки и библиографические списки.
   		\item Так как исходный документ содержит просто текст, с помощью программных средств на любом языке программирования можно создать таблицы, рисунки, формулы и т.д.
	\end{itemize}	
	\textit{Недостатки:}
	\begin{itemize}
		\item Во время редактирования документа нельзя (обычно) увидеть его окончательный вариант.
   		\item Необходимо знать нужные команды разметки \LaTeX.
   		\item Иногда сложно получить требуемый вид документа.
	\end{itemize}
	Документ \LaTeX — обычный текстовый файл, в котором указано содержание документа вместе с дополнительной разметкой. При обработке исходного файла макросами \LaTeX можно получить документ в разных форматах. Изначально \LaTeX поддерживает форматы DVI и PDF, но при использовании другого ПО можно легко получить PostScript, PNG, JPG и т.д.

		\subsubsection{Выбор инструмента редактирования}
			В ходе изучения всех возможных вариантов работа с \LaTeX ~для создания данного отчета, была выбрана программа Textmaker
			
			Выбор Textmaker'а обусловлен следующими его особенностями:
			\begin{itemize} 
	    		\item	Автоматическая подсветка синтаксиса
	    		\item	Функция автодополнения команд \LaTeX
	    		\item	Сокрытие блоков кода 
	    		\item	Быстрая навигация по структуре документа
	    		\item	Указание на строку с ошибкой, для быстрой отладки
	    		\item	Интегрированный просмотр PDF
			\end{itemize} 
	\newpage
	\subsection{Система контроля версий Git}
		\subsubsection{Описание}
		Система управления версиями — программное обеспечение для облегчения работы с изменяющейся информацией. Система управления версиями позволяет хранить несколько версий одного и того же документа, при необходимости возвращаться к более ранним версиям, определять, кто и когда сделал то или иное изменение, и многое другое.\\
		
		Git — это гибкая, распределённая система управления версиями. Проект был создан Линусом Торвальдсом для управления разработкой ядра Linux, первая версия выпущена 7 апреля 2005 года. На сегодняшний день его поддерживает Джунио Хамано. Программа является свободной и выпущена под лицензией GNU GPL версии 2.\\ 
		
	У каждого разработчика, использующего Git, есть свой локальный репозиторий, позволяющий локально управлять версиями. Затем, сохраненными в локальный репозиторий данными, можно обмениваться с другими пользователями. Часто при работе с Git создают центральный репозиторий, с которым остальные разработчики синхронизируются. В этом случае все участники проекта ведут свои локальны разработки и беспрепятственно скачивают обновления из центрального репозитория. Когда необходимые работы отдельными участниками проекта выполнены и отлажены, они, после удостоверения владельцем центрального репозитория в корректности и актуальности проделанной работы, загружают свои изменения в центральный репозиторий. Работа над версиями проекта в Git может вестись в нескольких ветках, которые затем могут с легкостью полностью или частично объединяться, уничтожаться, откатываться и разрастаться во все новые и новые ветки проекта.\\
		\subsubsection{Сравнению с другими системами контроля версий}
		 	\textit{Достоинства:}
				\begin{itemize}
\itemНадежная система сравнения ревизий и проверки корректности данных, основанные на алгоритме хеширования Secure Hash Algorithm 1.
\itemГибкая система ветвления проектов и слияния веток между собой.
\itemНаличие локального репозитория, содержащего полную информацию обо всех изменениях, позволяет вести полноценный локальный контроль версий и заливать в главный репозиторий только полностью прошедшие проверку изменения.
\itemВысокая производительность и скорость работы.
\itemУдобный и интуитивно понятный набор команд.
\itemМножество графических оболочек, позволяющих быстро и качественно вести работы с Git’ом.
\itemВозможность делать контрольные точки, в которых данные сохраняются без дельта компрессии, а полностью. Это позволяет уменьшить скорость восстановления данных, так как за основу берется ближайшая контрольная точка, и восстановление идет от нее. Если бы контрольные точки отсутствовали, то восстановление больших проектов могло бы занимать часы.
\itemШирокая распространенность, легкая доступность и качественная документация.
\itemГибкость системы позволяет удобно ее настраивать и даже создавать специализированные контроля системы или пользовательские интерфейсы на базе git.
\itemУниверсальный сетевой доступ с использованием протоколов http, ftp, rsync, ssh и др.\\
				\end{itemize}
			\textit{Недостатки:}
				\begin{itemize}
\item Unix – ориентированность. На данный момент отсутствует зрелая реализация Git, совместимая с другими операционными системами.
\itemВозможные (но чрезвычайно низкие) совпадения хеш - кода отличных по содержанию ревизий.
\itemНе отслеживается изменение отдельных файлов, а только всего проекта целиком, что может быть неудобно при работе с большими проектами, содержащими множество несвязных файлов.
\itemПри начальном (первом) создании репозитория и синхронизации его с другими разработчиками, потребуется достаточно длительное время для скачивания данных, особенно, если проект большой, так как требуется скопировать на локальный компьютер весь репозиторий. 
				\end{itemize}
		\subsubsection{Основные команды}
		\textit{add:} Добавляет содержимое рабочей директории в индекс для последующего коммита.\\
		
		\textit{status:} Показывает состояния файлов в рабочей директории и индексе: какие файлы изменены, но не добавлены в индекс; какие ожидают коммита в индексе. Вдобавок к этому выводятся подсказки о том, как изменить состояние файлов.\\
		
		\textit{diff:} Используется для вычисления разницы между любыми двумя Git деревьями.\\ 
		
		\textit{difftool:} Запускает внешнюю утилиту сравнения для показа различий в двух деревьях, на случай если вы хотите использовать что-либо отличное от встроенного просмотрщика git diff.\\
		
		\textit{commit:} Берёт все данные, добавленные в индекс с помощью git add, и сохраняет их слепок во внутренней базе данных, а затем сдвигает указатель текущей ветки на этот слепок.\\
		
		\textit{reset:} Используется в основном для отмены изменений. Она изменяет указатель HEAD и, опционально, состояние индекса.\\
		
		\textit{rm:} Используется в Git для удаления файлов из индекса и рабочей директории. Она похожа на git add с тем лишь исключением, что она удаляет, а не добавляет файлы для следующего коммита.\\
		
		\textit{mv:} Удобный способ переместить файл, а затем выполнить git add для нового файла и git rm для старого.\\
		
		\textit{clean:}Удаление мусора из рабочей директории. Это могут быть результаты сборки проекта или файлы конфликтов слияний.
		
		\subsubsection{GitHub}
		GitHub — крупнейший веб-сервис для хостинга IT-проектов и их совместной разработки. Основан на системе контроля версий Git и разработан на Ruby on Rails и Erlang компанией GitHub, Inc (ранее Logical Awesome).\\
		
		Для выполнения практической работы создан репозиторий в аккаунте \href{https://github.com/RandomRuslan/SMP_Practice}{RandomRuslan} на GitHub'е.\\
\newpage
\section{Исследование обучающих материалов}
	\subsection{Описание}
	\subsection{Критерий сравнения}
	\subsection{Иностранные материалы}
	\subsection{Российские материалы}
\newpage
\section{Используемая литература}
	\begin{enumerate}
		\item \href{https://ru.wikibooks.org/wiki/LaTeX}{https://ru.wikibooks.org/wiki/LaTeX}
%		\item http://all-ht.ru/inf/prog/p_0_1.html
		\item \href{https://git-scm.com/book/ru/v2}{https://git-scm.com/book/ru/v2}
		
	\end{enumerate}
\end{document}
